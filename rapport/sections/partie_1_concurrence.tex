
\subsection{Concurrence entre les firmes existantes}

\subsubsection{Définition du marché}

Le marché concerne le fret aérien qui désignera dans la suite tous les biens y compris le fret postal à l’exception des bagages.(IATA)

Cette étude porte sur les sociétés de Fret Aérien Civil Traditionnelles (FACT) qui gèrent leur propre flotte aérienne pour le transport de biens. Ce qui exclus le fret militaire ainsi que les logisticiens, entreprises de fret "virtuelles", qui font souvent du point à point par le biais de location de charters ou d'achat d'emplacements aux compagnies de fret traditionnlles. Les compagnies virtuelles constituent donc à la fois des clients pour le Fret Aérien Civil Traditionnel (FACT) qui permet à ces dernières d'optimiser le remplissage de leurs avions-cargo, mais aussi une forme de concurrence puisque le fret transporté au nom du logisticien est "perdu" pour tous les transporteurs classiques, y compris celui qui assure réellement le service. 



\subsubsection{Description des produits concernés}
Pour évaluer la concurrence, il convient d'étudier en premier lieu, le type de produits transportés. En effet, le pétrole ou le gaz, ne sont pas en général, transportés par avions ; sur ces matières premières, le fret aérien ne concurrence pas les secteurs maritimes, routiers, ferrovières, fluviaux ou par oléoduc/gazoduc. 

Le fret aérien, pour être compétitif, concerne donc les produits à haut rapport valeur-poids, souvent aussi à forte valeur ajoutée, ou à forte contrainte temporelle :


\begin{itemize}
	\item Electronique,
	\item Produits périssables : fleurs, fruits, ...
	\item Produits urgents ou à finalités humanitaires.
\end{itemize}



\subsubsection{Taxonomie des acteurs du transport du fret}

On distingue 3 types de sociétés de transport qui exhibent des structures de coûts, des ca\-racté\-risti\-ques opérationnelles et une répartition spatiale de l'offre et de la demande distinctes. En particulier, le transport en soute peut-être facturé au coût marginal car les coûts directs d'exploitation du vol sont imputés aux passagers. Le fret en soute possède donc un avantage concurrentiel sur le tout-cargo. Trois solutions possibles à ce problème, la régulation des prix pour protéger les freighters, le ré-ajustement de la flotte pour les compagnies mixtes qui revendent leur cargo, voire constituent des filiales disjointes, ou la différentiation en offrant des services spécifiques à leur clients pour les entreprises tout cargo qui ne sont pas astreintes à opérer sur des aéroports destinés au traffic passager ; il s'agit ici de tirer partie
de la souplesse dans la localisation et la topologie du réseau mondial de hubs de fret.
Les compagnies mixtes sont soumises à une barrière de sortie du secteur du fret "tout-cargo" beaucoup plus souple et rentable. Les tout-cargo, quant à elles, n'ont d'autres choix que le rachat par un concurrent ou nouvel entrant, de mettre la clé sous la porte ou de fusionner. Des alliances permettent des économies d'échelles et des synergies bénéfiques en terme de qualité et versatilité des services proposés au clients.



Les \textit {"freighters"} qui possèdent une flotte d'avions cargo dédiés qui vont du gros porteurs B747-8F de rayon d'action 8000 km et de capacité 140 tonnes, au Beluga AirBus A300 reconverti jusqu'au simple twin turboprop Cessna super cargomaster (RA : 1700 km, capacité fret : 1.8 tonnes) Ainsi, un transporteur tout\-cargo de fret classique comme Cargolux a très peu de frais de personnel et de frais commerciaux. En revanche, les coûts liés aux avions, aux redevances aéronautiques et aux frais d’escale sont élevés. En 2014, la part de marché en valeur, en revenu tonne kilomètre global (RTK), pour ce type de compagnies se monte à 56 \%. La tendance depuis 2000 est à la baisse au profit du transport en soute avec probablement, une part de marché en quantité qui ne dépasse pas désormais les 30 \%. L'adverbe \textit{probablement} souligne ici la difficulté d'obtention et d'interprétation des données aéronautiques. Les freighters sont en outre confrontées au problème de la \textit{bi-directionnalité} c'est à dire au retour à vide après livraison, spécifité qui n'existe pas dans le transport de passagers. 
	
Les \textbf{compagnies mixtes} telles que Lufthansa, Air France-KML utiliseront soit le transport en soute dans leurs avions-passagers, soit des avions cargo combinés, i.e. avions configurés de manière permanente pour le transport de fret et de PAX, soit des avions reconfigurables rapidement pour les deux types de transport. Pour illustrer le transport en soute, on note qu'un moyen-courrier assurant un vol avec 200 passagers représente un revenu d’environ US\$100 000, auxquels s’ajouteront quelque US\$13000 en fonction du taux de remplissage des soutes de l’avion. Un moyen-courrier de la catégorie A330/B767 permet d’embarquer environ 10 tonnes de fret hors bagages. En 2015, la valeur moyenne de chaque kilo transporté par avion s’établit à US\$127, contre US\$1,10 pour le maritime.
Certaines compagnies mixtes possèdent des filiales spécialisées dans le cargo. On peut citer British Airways World Cargo classée 12ème en terme de freight tonne-kilomètres (FTK). Mais même dans ce contexte, British Airways
va cessé l'exploitation de ses Boeing 747-8F. Source : \url{http://www.economist.com/node/21600946#print}
	
Les \textbf{intégrateurs}\label{integrateur} comme FedEx, UPS, TNT, qui font du point à point via des hubs dédiés, sont des entreprises avec des frais de personnel plus élevés et des coûts avion moindres grâce à une flotte d'appareils d’occasion convertis en cargo : A300-600, A310. \cite{lantenne}


\subsubsection{Description du marché}
Le fret aérien génère un chiffre d’affaires annuel évalué par l’Association internationale du transport aérien (IATA) à 47,8 milliards de dollars en 2016. Les avions ont embarqué 53,9 millions de tonnes de fret en 2016. L’aérien ne représente qu’un faible pourcentage du volume du fret (environ 5 \%),
mais environ 35 à 40 \% en valeur. \cite[lantenne]. Ce secteur demeure toutefois une activité minoritaire au sein des compagnies aériennes, le transport de passagers produisant 80 à 85 \% de ses recettes. Le transport maritime constitue le principal concurrent de l’aérien. Du fait de son coût très faible, les entreprises optimisent leur logistique pour rendre comptatible les délais de transport par mer compatibles avec leur activité. Les autres modes de fret, a contrario, se présentent parfois comme complémentaires de l'aérien : fret camionné de Toulouse au hub de Roissy-CDG par exemple.


En terme de taille de marché, on comptait en 2015 environ 200 firmes spécialisées dans le cargo, y compris les intégrateurs définis § \ref{integrateur}, auquel s'ajoute un nombre approximatif de 1100 compagnies aériennes dédiées au transport de passagers et appelées de plus en plus à alimenter l'offre en fret AFTK (Available Fret Konne-Km) ; les "Wide-Body" aux capacités en soute augmentée n'y sont pas étrangers.

Pour donner une idée, de la répartition des acteurs sur le marché, à défaut de pouvoir obtenir des données plus précises, on remarque en 2014 la prédominance des intégrateurs FedEx (7 Millions de tonnes transportés) et UPS (4 Mt) et des compagnies asiatiques et du moyen-orient, hub des pays du golf : Emirates, Korean Air, Cathay Pacific Airways à Hong-Kong. Sources : \url{http://aircargoworld.com/freight-50-the-top-50-cargo-carriers/} 

L'évolution du marché est caractérisé par une croissance molle au niveau global et un contraste géographique marqué dû notamment aux hub du moyen-orient et d'Asie. Sans développer davantage, on peut noter que la croissances du fret et du traffic passagers dans ces régions induit un accroissement du surplus de l'offre en fret (AFTK). Chaque mise en service d'un Boeing 777 ajoute 25 tonnes de fret 
à l'offre déjà surdimensionnée grévant ainsi la santé global du secteur. Sources : \url{http://www.economist.com/node/21600946#print}




