
\section{Concurrence entre les firmes existantes}

\subsection{Définition du marché}

Le marché concerne le fret aérien qui désignera dans la suite tous les biens y compris le fret postal à l’exception des bagages.(IATA)

Cette étude porte sur les sociétés de Fret Aérien Civil Traditionnelles (FACT) qui gèrent leur propre flotte aérienne pour le transport de biens. Ce qui exclus le fret militaire ainsi que les logisticiens, entreprises de fret "virtuelles", qui font souvent du point à point par le biais de location de charters ou d'achat d'emplacements aux compagnies de fret traditionnlles. Les compagnies virtuelles constituent donc à la fois des clients pour le Fret Aérien Civil Traditionnel (FACT) qui permet à ces dernières d'optimiser le remplissage de leurs avions-cargo, mais aussi une forme de concurrence puisque le fret transporté au nom du logisticien est "perdu" pour tous les transporteurs classiques, y compris celui qui assure réellement le service. 

\subsection{Description du marché}
Le fret aérien génère un chiffre d’affaires annuel évalué par l’Association internationale du transport aérien (Iata) à 47,8 milliards de dollars en 2016. Les avions ont embarqué 53,9 millions de tonnes de fret en 2016. L’aérien ne représente qu’un faible pourcentage du volume du fret (environ 5 \%),
mais environ 35 à 40 \% en valeur. \cite[lantenne]. Ce secteur demeure toutefois une activité minoritaire au sein des compagnies aériennes, le transport de passagers produisant 80 à 85 \% de ses recettes. Le transport maritime constitue le principal concurrent de l’aérien. Du fait de son coût très faible, les entreprises optimise
leur logistique pour rendre comptatible les délais de transport par mer compatibles avec leur activité. Les autres modes de fret, a contrario, se présentent parfois comme complémtaires de l'aérien : fret camionné de Toulouse au hub de Roissy-CDG 



\subsection{Description des produits concernés}
Pour évaluer la concurrence, il convient d'étudier en premier lieu, le type de produits transportés. En effet, le pétrole ou le gaz, ne sont pas en général, transportés par avions ; sur ces matières premières, le fret aérien ne concurrence pas les secteurs maritimes, routiers, ferrovière, fluviaux ou par oléoduc/gazoduc. 

Le fret aérien, pour être compétitif, concerne donc les produits à haut rapport valeur-poids, souvent aussi à forte valeur ajoutée, ou à forte contrainte temporelle :


\begin{itemize}
	\item Electronique,
	\item Produits périssables : fleurs, fruits, ...
	\item Produits urgents ou à finalités humanitaires.
\end{itemize}









\subsection{Taxonomie des acteurs du transport du fret}

On distingue 3 types de sociétés de transport qui exhibent des structures de coûts, des caractéristiques opérationnelles et une répartition spatiale de l'offre et de la demande distinctes. En particulier, le transport en soute peut-être facturé au coût marginal car les coûts directs d'exploitation du vol sont imputés aux passagers. Le fret en soute possède un avantage concurrentiel sur le tout-cargo. Trois solutions possibles à ce problème, la régulation des prix, le ré-ajustement de la flotte pour les compagnies mixtes qui revendent leur cargo ou la différentiation en offrant des services spécifique pour les entreprises tout cargo qui ne sont pas astreintes à opérer sur des aéroports destinés au traffic passager. 

Pb pour rentabiliser le retour contrairement au transport passager.
solution : route multisecteur avec passage par plusieurs aéroports
mais solution non viable pour les produits périssables.


\begin{itemize}
	\item Les \textit {"freighters"} qui possèdent une flotte d'avions cargo dédiés qui vont du gros porteurs B747-8F de rayon d'action 8000 km et de capacité 140 tonnes, au Beluga AirBus A300 reconverti jusqu'au simple twin turboprop Cessna super cargomaster (RA : 1700 km, capacité fret : 1.8 tonnes) Ainsi, un transporteur tout\-cargo de fret classique comme Cargolux a très peu de frais de personnel et de frais commerciaux. En revanche, les coûts liés aux avions, aux redevances aéronautiques et aux frais d’escale sont élevés. En 2014, la part de marché en valeur, en revenu tonne kilomètre global (RTK), pour ce type de compagnies se monte à 56 \%. La tendance depuis 2000 est à la baisse au profit du transport en soute avec probablement, une part de marché en quantité qui ne dépasse pas les 30 \%. Les freighters sont en outre confrontées au problème de la bi-directionnalité c'est à dire au retour à vide après livraison. 
	
	\item Les compagnies mixtes telles que Lufthansa, Air France-KML utiliseront soit le transport en soute dans leurs avions-passagers, soit des avions cargo combinés, i.e. avions configurés de manière permanente pour le transport de fret et de PAX, soit des avions reconfigurables rapidement pour les deux types de transport. Un moyen-courrier assurant un vol avec 200 passagers représente un revenu d’environ US\$100.000, auxquels s’ajouteront quelque US\$13.000 en fonction du taux de remplissage des soutes de l’avion. Un moyen-courrier de la catégorie A330/B767 permet d’embarquer environ 10 tonnes de fret hors bagages. En 2015, la valeur moyenne de chaque kilo trans-
	porté par avion s’établit à US\$127 , contre US\$1,10 pour le maritime. 
	
	\item Les intégrateurs comme FedEx, UPS, TNT, qui font du point à point via des hubs dédiés, sont des entreprises avec des frais de personnel plus élevés et des coûts avion moindres grâce à une flotte d'appareils d’occasion convertis en cargo : A300-600, A310.Source : \url{http://www.lantenne.com}
\end{itemize}

