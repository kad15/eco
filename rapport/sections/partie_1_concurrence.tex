
\subsection{Concurrence entre les firmes existantes}

\subsubsection{Définition du marché}

Le marché concerne le fret aérien qui désignera dans la suite tous les biens y compris le fret postal à l’exception des bagages. (IATA)

Cette étude porte sur les sociétés de Fret Aérien Civil Traditionnelles (FACT) qui gèrent leur propre flotte aérienne pour le transport de biens. Ce qui exclut le fret militaire ainsi que les logisticiens, entreprises de fret "virtuelles", qui font souvent du point à point par le biais de location de charters ou d'achat d'emplacements aux compagnies de fret traditionnelles. Les compagnies virtuelles constituent donc à la fois des clients pour les acteurs de Fret Aérien Civil Traditionnel (FACT) en leur permettant d'optimiser le remplissage de leurs avions-cargo, mais aussi une forme de concurrence puisque le fret transporté au nom du logisticien est "perdu" pour tous les transporteurs classiques, y compris celui qui assure réellement le service.



\subsubsection{Description des produits concernés}
\label{produits}
Pour évaluer la concurrence, il convient d'étudier en premier lieu le type de produits transportés. En effet, le pétrole ou le gaz, ne sont pas en général, transportés par avions ; sur ces matières premières, le fret aérien ne concurrence pas les secteurs maritimes, routiers, ferroviaires, fluviaux ou par oléoduc/gazoduc. 

Le fret aérien, pour être compétitif, concerne donc les produits à haut rapport valeur-poids, souvent aussi à forte valeur ajoutée, ou à forte contrainte temporelle :


\begin{itemize}
	\item Électronique,
	\item Produits périssables : fleurs, fruits...
	\item Produits urgents (colis express...) ou à finalités humanitaires.
\end{itemize}



\subsubsection{Taxonomie des acteurs du transport du fret}
\label{taxonomie}
On distingue 3 types de sociétés de transport qui exhibent des structures de coûts, des caractéristiques opérationnelles et une répartition spatiale de l'offre et de la demande distinctes.\\


Les \textbf{compagnies de cargo} qui ont pour cœur de métier le seul transport aérien de fret, principalement sur les liaisons long-courriers ou transatlantiques. Ces compagnies possèdent une flotte d'avions tout-cargo dédiés, qui vont du gros porteur B747-8F de rayon d'action 8000 km et de capacité 140 tonnes, au Beluga – Airbus A300 reconverti – jusqu'au simple twin turboprop Cessna super cargomaster (RA : 1700 km, capacité de fret : 1.8 tonnes). Un transporteur de fret classique comme Cargolux a très peu de frais de personnel et de frais commerciaux. En revanche, les coûts liés aux avions, aux redevances aéronautiques et aux frais d’escale sont élevés. Seuls 10 à 15 \% du trafic mondial de fret aérien est réalisé par ce type de compagnies. \cite{popescu}
	
Les \textbf{compagnies mixtes} telles que Lufthansa, Air France-KML ou encore Emirates et Korean Air utiliseront soit le transport en soute dans leurs avions-passagers, soit des avions cargo combinés, i.e. des avions configurés de manière permanente pour le transport de fret et de PAX ou des avions reconfigurables rapidement pour les deux types de transport. Pour illustrer le transport en soute, on note qu'un moyen-courrier assurant un vol avec 200 passagers représente un revenu d’environ US\$100 000, auxquels s’ajouteront quelque US\$13 000 en fonction du taux de remplissage des soutes de l’avion. Un moyen-courrier de la catégorie A330/B767 permet d’embarquer environ 10 tonnes de fret hors bagages. En 2015, la valeur moyenne de chaque kilo transporté par avion s’établit à US\$127, contre US\$1,10 pour le maritime.
Certaines compagnies mixtes possèdent aussi des filiales spécialisées dans le cargo. On peut citer par exemple Emirates SkyCargo, classée 3e en terme de fret tonne-kilomètres (FTK) ou British Airways World Cargo classée 12e. Mais même dans ce contexte, British Airways va cesser l'exploitation de ses Boeing 747-8F dédiés pour se recentrer sur les opportunités qu'offre le cargo en soute. \cite{theEconomist01}

Les \textbf{intégrateurs}\label{integrateurs} comme FedEx, UPS, TNT ou DHL, qui font du point à point avec des avions tout-cargo via des hubs dédiés, sont des entreprises avec des frais de personnel plus élevés et des coûts avion moindres grâce à une flotte d'appareils d’occasion convertis en cargo : A300-600, A310. \cite{lantenne} Dans leur logique porte-à-porte, ces intégrateurs opèrent une chaîne logistique multimodale complète en disposant également de leurs propres entrepôts, camions et maillages routiers.\\

On remarque que le transport de fret en soute peut être facturé au coût marginal car les coûts directs d'exploitation du vol sont imputés aux passagers, ce qui lui offre un avantage concurrentiel sur le tout-cargo. Trois solutions possibles à ce problème : la régulation des prix pour protéger les tout-cargo, le ré-ajustement de la flotte pour les compagnies mixtes qui revendent leur cargo, voire constituent des filiales disjointes, ou la différentiation en offrant des services spécifiques. Ce dernier point vaut notamment pour les entreprises tout-cargo qui ne sont pas astreintes à opérer sur des aéroports destinés au trafic passager ; il s'agit ici de tirer partie
de la souplesse dans la localisation et la topologie du réseau mondial de hubs de fret. Dans ce sens, nous verrons que les intégrateurs ont su s'imposer sur le marché en apportant une solution clé en main.


Enfin, les compagnies mixtes sont soumises à une barrière de sortie du secteur du fret beaucoup plus souple et rentable. Les tout-cargo, quant à elles, n'ont d'autres choix que le rachat par un concurrent ou nouvel entrant, de mettre la clé sous la porte, d'évoluer ou de fusionner. À l'instar du transport de passagers, des alliances permettent des économies d'échelles et des synergies bénéfiques en terme de qualité et versatilité des services proposés au clients. Des ententes illégales sur le prix du fret ont été révélées il y a quelques années, entre Qantas Airways et plusieurs autres compagnies ou encore Air France-KLM, Cathay Pacific et SAS Cargo...\cite{popescu}


\subsubsection{Description du marché}
Le fret aérien génère un chiffre d’affaires annuel évalué par l’Association internationale du transport aérien (IATA) à 47,8 milliards de dollars en 2016. Les avions ont embarqué 53,9 millions de tonnes de fret en 2016. L’aérien ne représente qu’un faible pourcentage du volume du fret (environ 5 \%),
mais environ 35 à 40 \% en valeur. \cite{lantenne}. Ce secteur demeure toutefois une activité minoritaire au sein des compagnies aériennes, le transport de passagers produisant 80 à 85 \% de ses recettes. Le transport maritime constitue le principal concurrent de l’aérien. Du fait de son coût très faible, les entreprises optimisent leur logistique pour rendre compatibles les délais de transport avec leur activité. Les autres modes de fret, a contrario, se présentent parfois comme complémentaires de l'aérien : fret camionné de Toulouse au hub de Roissy-CDG par exemple.

En terme de taille de marché, on comptait en 2015 environ 200 firmes spécialisées dans le cargo, y compris les intégrateurs définis § \ref{taxonomie}, auquel s'ajoute un nombre approximatif de 1100 compagnies aériennes dédiées au transport de passagers et appelées de plus en plus à alimenter l'offre en fret AFTK (Available Fret Tonne-Km) ; les \textit{wide-body} aux capacités en soute augmentée n'y sont pas étrangers.

En 2014, la part de marché en valeur, en revenu tonne kilomètre global (RTK), pour le tout-cargo (compagnies de cargo et intégrateurs) se monte à 56 \%. La tendance depuis 2000 est à la baisse au profit des compagnies mixtes avec transport en soute, avec probablement une part de marché en quantité qui ne dépasse pas désormais les 30 \%. L'adverbe \textit{probablement} souligne ici la difficulté d'obtention et d'interprétation des données aéronautiques. Le tout-cargo est en outre confronté au problème de la \textit{bi-directionnalité} c'est-à-dire au retour à vide après livraison, spécificité qui n'existe pas dans le transport de passagers, et que seules des optimisations en termes de flux et de placement des hubs permettent de soulager.

Pour donner une idée de la répartition des acteurs sur le marché, à défaut de pouvoir obtenir des données plus précises, on remarque en 2014 la prédominance des intégrateurs FedEx (7 millions de tonnes transportés) et UPS (4 Mt) et des compagnies asiatiques et du Moyen-Orient, hub des pays du Golfe : Emirates, Korean Air, Cathay Pacific Airways à Hong-Kong. \cite{top50}

L'évolution du marché est caractérisée par une croissance molle au niveau global et un contraste géographique marqué dû notamment aux hubs du Moyen-Orient et d'Asie. Sans développer davantage, on peut noter que la croissances du fret et du trafic passagers dans ces régions induit un accroissement du surplus de l'offre en fret (AFTK). Chaque mise en service d'un Boeing 777 ajoute 25 tonnes de fret 
à l'offre déjà surdimensionnée grevant ainsi la santé globale du secteur. \cite{theEconomist01}


Au niveau stratégique, le secteur est confronté à divers facteurs : sécurité, environnement, politiques protectionnistes... On citera pour finir l'importance
de l'évolution de l'offre aéroportuaire pour éviter les problèmes de congestion.
Le climat peut en être une cause avec un nombre croissant d'aéroports inondables : 40 en Europe, 20 rien qu'en Norvège.