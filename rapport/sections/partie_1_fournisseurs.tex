\subsection{Fournisseurs}

Comme pour toutes les compagnies aériennes, la volatilité et le niveau des prix du pétrole est un souci permanent, compensable cependant par des investissements financiers et les solutions alternatives, réacteur à gaz ou à biocarburant, qui ne sont encore qu'à l'état de prototypes. Les contraintes réglementaires et environnementales imposent également de renouveler les flottes avec des avions plus récents.

Au niveau des constructeurs d'avions, les entreprises de fret sont, peu ou prou, confrontées au duopole Boeing - Airbus. Elles possèdent malgré tout d'une part un pouvoir de négociation proportionnel à leur taille et d'autre part la possibilité de se fournir sur le marché de l'occasion ou de faire appel à la location.

Le secteur du fret est un consommateur captif vis-à-vis des fournisseurs de services aéroportuaires. Les contraintes de volume et de poids imposent des équipements spécifiques permettant de garantir une manutention sûre et efficace, aussi bien pour les phases de (dé)chargement que par l'utilisation d'enceintes à rayons X adaptées. Les sociétés de fret sont aussi dépendantes des créneaux qui lui sont attribués et doivent faire face à la congestion du trafic en travaillant avec les différents acteurs aéroportuaires et de l'ATM. Ceux-ci peuvent alors être perçus comme des compléments auxquels il est difficile de s'abstraire totalement, bien qu'il soit possible dans un premier temps d'opérer tout ou partie d'une plate-forme aéroportuaire dédiée au fret.

Enfin, les compagnies de cargo doivent, pour rester attractives face aux compagnies mixtes et aux intégrateurs, contracter des accords avec des logisticiens pour proposer des services annexes tels que le transfert routier.