\subsection{Menaces d'éventuels entrants}


Les barrières à l'entrée semblent davantage relever de la présence de mastotondes tels que FedEx, UPS, TNT capables de pratiquer des prix bas dus aux économies d'échelles. En effet, la possibilité de louer ses avions plutôt que de les acheter
permet de créer des compagnie ariennes fussent-elles héphémères.

En outre, la sur-capacité actuelle de l'offre favorise plutôt la sortie que l'entrée et il semble qu'à l'avenir la demande va augmenter mais la capacité 
également.  

A priori, les nouveaux entrants éventuels se porteront plutôt sur des marchés de niche, à moins qu'ils ne surfent sur l'apport de nouvelles technologies telles que les drônes, les avions sans pilotes, le tout numérique pour réduire le coût des formalités et les accélérer, le centraliser. Mais alors les firmes en place ne sont-elles pas les mieux placées pour tirer partie de ces avancées quitte à racheter ses start-up ?


