\section{Rôle de l'ATM dans la stratégie des acteurs du fret aérien}

%Single European Sky ATM Research (SESAR)

Du fait des spécificités propres au secteur du fret aérien, l'ATM joue un rôle important dans l'organisation et la stratégie des différents acteurs. En effet, nous allons voir que le fait transporter des marchandises implique des problématiques parfois bien différentes à celles du transport de passagers.

\subsection{Un trafic réparti}

\subsubsection{Des vols ayant principalement lieu la nuit}

Comme le montre différents articles tels \cite{popescu}, le fret aérien a, dès ses origines, été principalement organisé avec des vols ayant lieu la nuit. En effet, cela permet plusieurs avantages du point de vue de l'organisation du trafic aérien. 

Tout d'abord, cette activité étant en grande partie organisée autour de grands hubs \cite{Walcott201764}, cela permet aux marchandises de transiter le jour vers ces centres névralgiques pour ensuite être redistribuées dans le monde entier dans des avions chargés en toute fin de journée. 

Ensuite, les marchandises n'étant pas soumis aux mêmes contraintes temporelles que les passagers, leur transport dans des vols de nuit permet de répartir le trafic tout au long d'une journée : les heures de pointe sont alors consacrées aux vols passagers et les créneaux disponibles la nuit aux vols cargo. Cela permet donc de ne pas rajouter de trafic lors de périodes de congestion puisque les marchandises peuvent partir en plein milieu de la nuit.

Une organisation de nuit peut également permettre à des entreprise de transport de fret de réduire certains coûts et d'assouplir leur organisation. En effet, ces vols ayant lieu hors période de congestion, les redevances aéroportuaires et de contrôles peuvent être moins élevées qu'en plein pic de trafic. De même, la demande étant moindre l'allocation de créneau s'en trouve grandement facilité.

Enfin, cette organisation autour de vols de nuit permet à certains acteurs de diversifier leur activité. On peut citer en France l'exemple de ASL Airlines France (anciennement Europe Airpost) qui possède plusieurs B737-300 \textit{Quick-Change} lui permettant d'effectuer des vols passagers le jour et des vols de fret la nuit.\\

On voit donc que les différents acteurs du transport aérien de fret prennent en considérations un certain nombre de problématiques liées à l'organisation générale du trafic aérien et décident alors d'organiser principalement leur activité la nuit. Nous allons cependant voir que d'autres problématiques ATM peuvent porter atteinte à ce mode de fonctionnement.

\subsubsection{Un trafic de nuit remis en cause}

\subsection{Un réseau fortement maillé}

