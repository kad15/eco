
\section{Rôle de l'ATM dans le futur par rapport aux stratégies des acteurs }

L'\textit{Air Traffic Management} est un domaine en perpétuelle évolution, tout particulièrement ces dernières années en Europe. En effet, il s'agit d'un secteur où les procédures et les outils doivent évoluer et s'adapter au trafic afin d'assurer une meilleure gestion du trafic tout en gardant un niveau de sécurité élevé. Nous allons voir un certain nombre d'évolutions qui pourraient avoir un impact sur les acteurs du fret aérien.

\subsection{L'opportunité du programme SESAR}

Comme le montre \cite{52008DC0750}, le programme \textit{Single European Sky ATM Research} (SESAR) mis en place par la Commission Européenne affiche d'ambitieux objectifs tels que :
\begin{itemize}
\item une réduction de moitié des coûts de contrôle aérien;
\item une réduction de 10\% de l'impact sur l'environnement;
\item une division du risque d'accident par 10;
\item un triplement de la capacité de l'espace aérien.
\end{itemize}

Dans ce contexte, les entreprises de fret aérien, au même titre que l'ensemble des compagnies aériennes volant en Europe, vont voir apparaitre la possibilité d'augmenter leurs capacités et de diminuer leurs coûts. Ainsi, si les échanges commerciaux continuent à croitre, notamment avec l'Asie, ces entreprises de transport pourront se développer sans contraintes immédiates dues à la gestion du trafic aérien.

Ces objectifs pourront être rendus possibles par le progrès technique (moteurs plus économiques et moins bruyants), la généralisation de procédures efficaces (descente continue) et la refonte du ciel européen qui souffre aujourd'hui d'une coûteuse segmentation et d'un manque d'harmonisation et de standardisation.\\

On voit donc en quoi les travaux actuels de modernisation de l'ATM permettent d'ouvrir un certain nombre de perspectives aux entreprises de transport aérien en général et, par conséquent, aux entreprises de transport aérien de fret en particulier.

\subsection{Introduction de nouveaux paradigmes}

Dans le cadre des réflexions relatives à la privatisation de certains ANSP (\textit{Air Navigation Service Provider}), des idées de services commerciaux que pourraient vendre ces entités ont émergé.

Par exemple, il serait envisageable de moduler le tarif de la redevance de contrôle en fonction du gain de temps ou du retard que serait prête à accepter une compagnie. Ainsi, une compagnie qui souhaiterait être prioritaire sur l'attribution d'un créneau paierait une redevance plus élevée qu'une compagnie prête à accepter un retard.

Or nous avons pu voir que l'une des spécificités du transport aérien de fret est que les marchandises sont, dans le cas général, résilientes face aux retards. Ainsi, grâce à ces services commerciaux liés à la gestion du trafic, les transporteurs aériens de fret pourraient optimiser leurs coûts dans certains cas.\\

On voit donc que de nouveaux paradigmes émergent au sein des ANSP et que ceux-ci peuvent impacter fortement la stratégie des transporteurs aérien de fret.

\subsection{Vers une extension du périmètre du fret}

%Etablissement de régulation lors du développement des drônes et avions sans pilotes : voilure mobile ou fixe. 

Comme le montre \cite{RePEc:eee:jaitra:v:61:y:2017:i:c:p:34-40}, de nouvelles entrées, certes limitées mais réelles, font leur apparition sur le marché du transport aérien de fret. Il s'agit principalement d'opérateurs de drones souhaitant réaliser un transport de marchandise par le biais de cette nouvelle technologie.

On peut citer notamment des entreprises comme La Poste \cite{gradt_2016} ou Amazon \cite{figaro_2016} qui tentent de mettre en place ce nouveau marché.

Si ce nouveau segment répond plutôt à la problématique du dernier kilomètre qui ne concerne pas tous les acteurs du transport aérien de fret, il est à noter que les évolutions de l'ATM vont jouer un rôle essentiel dans le développement de ce marché.

En effet, l'automatisation du transport aérien de marchandise pose un grand nombre de difficultés au regard de la gestion du trafic aérien dans certains espaces. De nouveaux concepts émergent alors : on parle ainsi de l'UTM (\textit{Unmanned Aircraft Systems} (UAS) \textit{Traffic Management}) au lieu de l'ATM pour désigner ces problématiques de gestion de trafic propres aux drones.\\ 

Ainsi, les évolutions technologiques dans les domaines de l'ATM, de l'UTM et des drones apporteront de nouvelles possibilités de développement aux acteurs du transport aérien de marchandises.





