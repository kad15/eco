\documentclass[a4paper,12pt]{article}


% package qui fournit \justify 
\usepackage[document]{ragged2e}

%pifont pour les puces de formes spéciales
\usepackage{pifont}

% césure exemple
% \hyphenation{an-ti-cons-ti-tu-tion-nel

\usepackage[utf8x]{inputenc}
\usepackage[T1]{fontenc}
\usepackage[frenchb]{babel} % If you write in French
%\usepackage[english]{babel} % If you write in English
\usepackage{lmodern} % Pour changer le pack de police
\renewcommand*\familydefault{\sfdefault}
\usepackage{makeidx}
\usepackage{amsthm}
\usepackage{amsmath}
\usepackage{amssymb}
\usepackage{mathrsfs}
\usepackage{stmaryrd}
\usepackage{geometry}
%\usepackage{graphicx}
\usepackage{graphbox}
\usepackage{supertabular}
\usepackage{tabularx}
\usepackage{longtable}
\usepackage{pdflscape}
\geometry{hmargin=2cm,vmargin=2cm}

\usepackage{booktabs}
\usepackage{tabularx}
\usepackage[table]{xcolor}
\usepackage{ltablex}
\usepackage{float}
\usepackage{url}

\usepackage{chngcntr}
\counterwithin*{footnote}{page}


\usepackage[titletoc,toc,title,page]{appendix}
\renewcommand{\appendixtocname}{Annexes}
\renewcommand{\appendixpagename}{Annexes}

\usepackage{standalone}
\usepackage{ifthen}
\usepackage{xstring}
\usepackage{calc}
\usepackage{pgfopts}
\usepackage{tikz}
\usetikzlibrary{positioning,shapes,shadows,arrows}

\usepackage{algpseudocode}
\usepackage{algorithm}
\makeatletter
\renewcommand{\ALG@name}{Algorithme}
\renewcommand{\listalgorithmname}{Table des algorithmes}

\newtheorem{theo}{Définition}[section]
\usepackage{mathtools, bm}
\usepackage{amssymb, bm}

\usepackage{hyperref}
\hypersetup{
    colorlinks=true,       % false: boxed links; true: colored links
    linkcolor=black,       % color of internal links
    citecolor=purple,       % color of links to bibliography
    urlcolor=blue          % color of external links
}
\usepackage{listings}
\definecolor{darkgreen}{rgb}{0, 0.6, 0}
\lstset{language = caml, frameround = fttt}

\lstset{upquote=true,
        columns=flexible,
        keepspaces=true,
        breaklines,
        breakindent=0pt,
        basicstyle=\ttfamily,
        breaklines=true,
        keywordstyle=\color{red},
        commentstyle=\color{darkgreen},
        tabsize=2,
        escapebegin=\color{gray},
}
\usepackage{blindtext}
\usepackage{enumitem} % pour changer les puces dans \itemize
\title{Distribution trouée Damn Vulnerable Web Application}
%\author{Florian Barbarin \\ Abdelkader Beldjilali \\ Alexis Letombe\\ \\ \\ %\\Encadrant : Cyril Allignol\\ \\ \\}

% \includegraphics[width=0.6\textwidth]{images/enac}\\[1cm]
\date{\today}

\makeindex
\def\siecle#1{\textsc{\romannumeral #1}\textsuperscript{e}}
\newcommand{\argmax}{\mathop{\mathrm{argmax}}\nolimits}
\newcommand{\pgcd}{\mathop{\mathrm{pgcd}}\nolimits}

\makeatletter
\renewcommand{\pod}[1]{\allowbreak\mathchoice
  {\if@display \mkern 18mu\else \mkern 8mu\fi (#1)}
  {\if@display \mkern 18mu\else \mkern 8mu\fi (#1)}
  {\mkern4mu(#1)}
  {\mkern4mu(#1)}
}


% pour avoir des figure encadrées
%\usepackage{float}
%\floatstyle{boxed} 
%\restylefloat{figure}




\begin{document}

% pour factoriser l'échelle des figures 
%utilisation scale=\scaledvwa au lieu de scale = 0.3 ... 
\newcommand{\scaledvwa}{0.4} 
\newcommand{\scaledvw}{0.3}
\newcommand{\scalekad}{0.45}

\input{page-de-couverture/page-de-couverture}%on créé la couverture

\pagebreak

\tableofcontents
\justify

\pagebreak

\section*{Introduction}
\addcontentsline{toc}{section}{Introduction}
%\markboth{Introduction}{} 
Comme TV5 monde en 2015, les pertes des entreprises victimes de cyberattaques, en forte augmentation, se comptent souvent en dizaines de millions d'euros. L'Open Web Application Security Project (OWASP : {\color{blue} \url{https://www.owasp.org}}), publie régulièrement la liste des 10 menaces les plus critiques qui concernent les applications web. Une manière de s'en prémunir consiste à pratiquer le hacking web éthique. C'est là qu'entre en scène l'outil DVWA. Damn Vulnerable Web App ({\color{blue} \url{http://www.dvwa.co.uk}}), est un environnement PHP qui permet de se former à la sécurité des sites web. Le hacker en herbe peut ainsi se former et tester légalement ses compétences sur une application hébergée en local. Une solution simple consiste à installer la distribution kali et le site web DVWA sur une machine virtuelle.

\begin{figure}[!h]
	\begin{center}
		\label{10_menaces}
		\includegraphics[scale=0.4]{images/10_menaces.png}
		\caption{Le top 10 des menaces web 2017 publiées par l'OWASP}
	\end{center}
\end{figure}



, 
Pour se former au hacking et donc à la sécurité des applications web, 

\newpage

\section{type d'attaque}

\subsection{Description de la vulnérabilité}

\subsection{Exploitation de la vulnérabilité}

\subsection{Contre-mesure}










\newpage

\section{File inclusion}

De nombreux langages de programmation permettent d'inclure des portions de code contenues dans d'autres fichiers que celui en cours d'exécution. Le mécanisme mis à disposition permet de recopier dans le script principal le code contenu dans un autre fichier. Cette procédure est transparente à l'œil de l'utilisateur et peut-être très avantageuse pour le développeur d'un site internet.

\begin{figure}[!h]
\begin{center}

\label{inclusion}
\includegraphics[scale=1.2]{images/include.pdf}

\caption{Mécanisme d'inclusion d'un fichier}

\end{center}
\end{figure}

En effet, inclure du code contenu dans un autre fichier permet, entre autre, les deux utilisations suivantes :
\begin{itemize}
\item inclure des portions de code différentes en fonction de choix de l'utilisateur ou de l'environnement de ce dernier;
\item inclure des portions de code utilisées dans plusieurs scripts (par exemple une fonction de connexion à une base de données) afin de ne pas avoir à recopier les mêmes lignes à différents endroits et de ne modifier qu'un seul fichier en cas de modification de la fonction.
\end{itemize}


Nous voyons donc que le premier point ci-dessus permet d'obtenir une réelle adaptabilité du code alors que le second point donne la possibilité au développeur d'écrire du code concis et factorisé. Nous allons cependant voir que ce mécanisme n'est pas dépourvu de vulnérabilités.


\subsection{Description de la vulnérabilité}

La principale vulnérabilité connue dans le mécanisme que nous venons d'expliciter intervient lorsque l'inclusion d'un script est géré par une variable pouvant être contrôlé par un attaquant. On se retrouve alors plutôt dans le premier cas d'utilisation indiqué, c'est à dire inclure des portions de code différentes en fonction de choix de l'utilisateur ou de l'environnement de ce dernier. En effet, dans le second cas d'utilisation, l'inclusion du fichier est généralement écrit "en dur" dans le script principal et ne peut donc pas être facilement modifié par un attaquant.

\begin{figure}[!h]
\begin{center}

\label{inclusion}
\includegraphics[scale=1.4]{images/include_hacked.pdf}

\caption{Vulnérabilité d'inclusion d'un fichier}

\end{center}
\end{figure}


ceci est un test

\subsection{Exploitation de la vulnérabilité}

\subsection{Contre-mesure}

\section{File upload}

\subsection{Description de la vulnérabilité}

\subsection{Exploitation de la vulnérabilité}

\subsection{Contre-mesure}

\section{Insecure CAPTCHA}

\subsection{Description de la vulnérabilité}

\subsection{Exploitation de la vulnérabilité}

\subsection{Contre-mesure}










\newpage
\section{Injection SQL }


\subsection{Description}




L'injection SQL, en anglais SQL  injection, ou SQLi en abrégé, est une des attaques les plus dangereuses. Comme pour le Cross Site Scripting présenté dans la suite de ce document, il s'agit ici de tirer parti de l'absence de filtrage des entrées utilisateurs. Cette absence de contrôles permet à un hacker d'insérer du code qui sera interprété par l'analyseur cible, par exemple SQL.




\paragraph{}
Dans la cas particulier de l'injection SQL et du site DVWA, les requêtes SQL, imbriquées dans des scripts PHP qui récupèrent les saisies des utilisateurs, peuvent être détournées sur la base de la syntaxe du langage. 

\begin{figure}[!h]
	\begin{center}
		\includegraphics[scale=1]{images/bd.png}
		\caption{source : \url{https://xkcd.com/327/}}
	\end{center}
\end{figure}


\subsection{Exploitation}

\subsubsection{DVWA - Security level "low"}

La base de données contient 5 utilisateurs identifiés par les entiers de 1 à 5.
La mission proposée par le DVWA est de voler leurs mots de passe par injection SQL.

On règle la "DVWA security" sur low de manière à avoir un site web "damn vulnerable".
On saisit dans le champ User Id, une simple apostrophe i.e. '. Le site retourne le message
"\it {You have an error in your SQL syntax; check the manual that corresponds to your MariaDB server version for the right syntax to use near ''''' at line 1}"

\paragraph{}Cette simple apostrophe démontre que le site est vulnérable pour deux raisons : d'abord on sait que nos saisies sont interprétées directement par l'analyseur SQL ; elle ne sont pas filtrées. Ensuite parce que le site est "bavard".

\begin{figure}[!h]
	\begin{center}
		\label{}
		\includegraphics[scale=\scaledvwa]{images/sql/sqli1.png}
		\caption{Les saisies incorrectes donnent lieu à un message d'erreur SQL qui informe le hacker potentiel de l'absence de protection contre les SQLi. D'autres informations importantes sont dévoilées comme le type de base de donnée, ici MariaDB, version libre de MySql rachetée par Oracle.}
	\end{center}
\end{figure}

Un clic sur le bouton "View Source" affiche le code PHP de la page. On constate, en effet, qu'on peut saisir n'importe quoi dans le champ User Id, il sera transmis sans modification à la requête \$query via \$id.   

\begin{figure}[!h]
	\begin{center}
		\label{}
		\includegraphics[scale=0.8]{images/sql/code_low.png}
	\end{center}
\end{figure}



\begin{figure}[!h]
	\begin{center}
		\label{}
		\includegraphics[scale=\scaledvwa]{images/sql/sqli_low.png}
		\caption{Vol des mots de passe par l'injection SQLi sur site non protégé.}
	\end{center}
\end{figure}

L'injection SQL suivante :
{\color{red}

\begin{verbatim}
   ' UNION select password, last_name from users#
\end{verbatim}
}

%  a' UNION SELECT password,last_name from users;-- -&Submit=Submit
% url correspondante 
%  ?id='+UNION+SELECT+password%2Clast_name+from+users%23&Submit=Submit#

donne la requête suivante en remplaçant \$id dans le script PHP :

{\color{red}
\begin{verbatim}
$query  = "SELECT first_name, last_name FROM users WHERE user_id = '' UNION 
           SELECT password, last_name from users#   ';"; 
\end{verbatim}
}
Elle indique donc qu'on effectue l'union au sens mathématique des éléments recueillis par les deux requêtes. Le premier SELECT donne l'ensemble vide, le second donne tous les mots de passe et noms de la table users. On obtient donc les mots de passe faussement associés au champ "First name". Ces mots de passe sont cryptés. On pourra utiliser des techniques de révélation par ingénieurie sociales, recherche internet, force brute, dictionnaires ou rainbow tables. L'outil John the ripper peut entrer en action. On note que Smith est certainement aussi admin car ces deux noms d'utilisateurs ont le même hash donc le même mot de passe. Le hacker peut être confiant quant à la suite des opération car Smith ne semble pas être un adepte de la SSI. Le caractère \# en fin d'injection évite que PHP n'interprête la suite du code en particulier les caractères apostrophes et guillements qui donneraient une erreur SQL.

NB : C'est une technique répandue que de forcer l'analyseur SQL à ignorer le reste de la requête, en utilisant le symbole commentaire SQL double tiret - - les symboles de commentaires PHP dièse \#, {/* */}, {//} pour assurer que ce qui suit l'injection ne sera pas interprété.  



\paragraph{}
Un injection SQL peut aussi donner accès au système de fichier comme le montre l'injection ci-après :
\begin{verbatim} [ ' UNION ALL SELECT load_file('/etc/passwd'),null # ] \end{verbatim}

\begin{figure}[!h]
	\begin{center}
		\label{}
		\includegraphics[scale=\scaledvwa]{images/sql/sqli_low2.png}
		\caption{Récupération du fichier /etc/passwd via la commande load\_file : }
	\end{center}
\end{figure}




% ?id=a UNION SELECT password,last_name from users;-- -&Submit=Submit
% ?id=a %27UNION SELECT password,last_name from users#






\subsubsection{DVWA - Security level "Medium" et "High"}

Pour récupérer les mots de passe lorsque l'on règle le niveau de sécurité de DVWA sur "haut" et "medium", on utilise une combinaison des outils "Burp suite" et "sqlmap" fournis par kali linux. Le navigateur doit être configuré en utilisant ce proxy Burp suite, à savoir 127.0.0.1:8080. Cela permet l'interception des requêtes POST qui est alors copiée dans un fichier toto.txt utilisé ensuite dans la commande.

\begin{verbatim}
sqlmap -r ./toto.txt –dbs -D dvwa –dump all –os-shell
\end{verbatim}


\subsection{Contre-mesures}

Un certain nombre de règles permettent de se prémunir des attaques par injection de commandes SQL :
Pour se prémunir des injections SQL, on peut appliquer les principes suivants :
 \begin{itemize}[font=\color{magenta} \Large, label=\ding{43}]
	\item Vérifier le format des données saisies et notamment
	la présence de caractères spéciaux, 
	\item Éviter les comptes sans mot de passe,
	\item Ne pas afficher de messages d’erreur explicites affichant la requête ou une partie de la requête SQL,
	\item Supprimer les comptes utilisateurs non utilisés, notamment les comptes par défaut,
	\item Utiliser un firewall Applicatif de type mod\_security
    \item Désactiver l’option Load\_File.
\end{itemize}


Sur le plan pratique, un premier niveau de protection consiste à vouloir utiliser un outil comme mysqli\_real\_escape\_string() qui "échappe" les caractères indésirables : apostrophes, guillemets. Cette technique utilisée par DVWA - Medium level reste cependant vulnérable.

\begin{verbatim}
En effet, en utilisant l’HTTP URL Encoding, un espace devient 
un %20 dans l’URL, un  “!” devient un %21, une apostrophe %27, etc.
Cela nous permet donc ici de faire passer une guillemet ou une 
apostrophe de façon encodée pour ne pas qu’ils soient détectés 
et échappés par la fonction mysql_real_escape_string.
\end{verbatim}

Plus efficace est l'utilisation des "prepared statement". DVWA, level "impossible", utilise la classe \href{https://secure.php.net/manual/fr/class.pdostatement.php}{PDOStatment} pour préparer les requêtes et ainsi séparer le code des données.  


\begin{figure}[!h]
	\begin{center}
		\label{}
		\includegraphics[scale=0.6]{images/sql/sqli_impossible.png}
	\end{center}
\end{figure}








% % % % % % % % % % % % % % % % % % % % % % % % % % % % % % % % % % % % % % % % % % % % %



\section{Injection SQL aveugle}

\subsection{Description}

Les injections SQL aveugles, ou "blind SQL" en anglais, qu'on peut nommer BSQLi en abrégé,  sont des techniques utilisées lorsque le serveur n'est pas "bavard". Sur le plan du code PHP, il suffit de retirer la ligne de code suivante qui affiche dans une page html les erreurs SQL : 
\begin{verbatim}
or die('<pre>' . mysql_error() . '</pre>' );
\end{verbatim} 




\subsection{Exploitation}
Les attaques sur DVWA - level "low", se font progressivement. Elles nécessitent beaucoup plus de requêtes sur la base de donnée, ce qui pose un problème de discrétion pour l'attaquant. Une meilleure connaissance de SQL est aussi impérative. Ainsi, une attaque BSQLi manuelle effectuée sur le User ID de DVWA, visant à déterminer le nombre  de champs de la requête, en vue de faire une UNION, utilisera une injection "ORDER BY".

\begin{verbatim}
L'injection [ ' ORDER BY 1 # ], sans les crochets, ne renvoie rien,  par contre 
[ ' ORDER BY 3 # ] affiche "Unknown column '3' in 'order clause' ". 
La requête du script PHP utilise donc deux champs.
\end{verbatim} 

\begin{verbatim}
Ensuite, des injections [ ' UNION SELECT password, last_name FROM xxxx # ]
où xxxx désigne un nom de table évocant une liste d'utilisateurs ; on testera
utilisateurs, users, user, etc
\end{verbatim} 

\begin{verbatim}
Des injections [ ' UNION SELECT password, last_name FROM users 
WHERE LENGTH(password) = longeur # ] avec longueur = 1, 2, 3, ... 
vont permettre, itérativement, de connaître la taille du mot de passe.
\end{verbatim} 

\begin{verbatim}
Des injections [ ' UNION SELECT password, last_name FROM users 
WHERE LENGTH(password) = 8 AND SUBSTRING(password,1,1)='a' # ] 
testent si le mot de passe commence par "a".
\end{verbatim} 
\subsection{Contre-mesure}

On le voit ce type d'attaques peut devenir rapidement fastidieuses, voire impraticables. Une approche médiane consiste à écrire des scripts en python important des modules comme httplib et urllib.

Enfin, des produits sur étagères comme Burp suite/sqlmap ou, mieux,   \href{www.itsecteam.com}{havij} permettent de s'attaquer aux niveaux "Low", "Medium" "High" de manière plus efficace. Havij est cependant un outil windows ; Sous fedora 26, on doit installer une machine virtuelle windows ;  l'utilisation de wine est quant à elle souvent hasardeuse.
% % % % % % % % % % % % % % % % % % % % % % % % % % % % % % % % % % % % % % % % % % % % %


\section{Attaques XSS }Reflected XSS

\subsection{Description}
 désignées au choix par les acronymes CSS ou XSS et qui seront
Un site web qui fournit d'une part un service de recueille d'informations via des formulaires et d'autre part un service de publication sur le site de ces mêmes informations
\subsection{Exploitation}

\subsection{Contre-mesure}


\section{Attaques XSS enregistrées }

\subsection{Description}

\subsection{Exploitation}

\subsection{Contre-mesure}


















%
%
%@ARTICLE{1056964,
%	author={A. Shamir},
%	journal={IEEE Transactions on Information Theory},
%	title={A polynomial-time algorithm for breaking the basic Merkle-Hellman cryptosystem},
%	year={1984},
%	volume={30},
%	number={5},
%	pages={699-704},
%	keywords={Cryptography;Computer science;Graphics;Mathematics;Microcomputers;Polynomials;Protection;Public key;Public key cryptography;Security;Testing},
%	doi={10.1109/TIT.1984.1056964},
%	ISSN={0018-9448},
%	month={Sep},}
%
%
%@BOOK{MARTIN2004,
%	title = {Codage, cryptologie et applications},
%	publisher = {Presses polytechniques et universitaires romandes},
%	year = {2014},
%	author = {Martin, Bruno}
%}
%
%@Book{GJ1979,
%	author = "Michael R. Garey and David S. Johnson",
%	title = "{Computers and Intractability, A Guide to the Theory of {NP}-Completeness}",
%	publisher = "W.H. Freeman and Company",
%	year = 1979,
%	address = "New York"
%}
%
%@book{opac,
%	title = "Handbook of Applied Cryptography",
%	author = "Alfred J. Menezes and Paul C. Van Oorschot and Scott A. Vanstone",
%	publisher = "CRC Press",
%	address = "Boca Raton, London, New York",
%	url = "http://opac.inria.fr/record=b1092394",
%	isbn = "0-8493-8523-7",
%	year = 1996,
%	note = {\url{http://cacr.uwaterloo.ca/hac/}}
%}
%
%@ARTICLE{DEROFF,
%	author={J. Deroff and R. Goyat},
%	title={Le problème du sac à dos en cryptographie},
%	year={2007},
%	note = {\url{http://j.deroff.free.fr/rapportter.pdf}}
%}
%
%@Book{stinson1996,
%	author =       {Stinson, Douglas},
%	title =        {{Cryptographie -- Théorie et pratique}},
%	publisher =    {International Thomson Publishing},
%	year =         {1996},
%	note =         {Traduction de Serge Vaudenay},
%}
%
%@ARTICLE{COSTER,
%	author={Coster, Matthijs J. and Joux, Antoine  and La Macchia, Brian A. and Odlyzko, Andrew M. and Schnorr, Claus-Peter and Stern, Jacques},
%	title={Improved low-density subset sum algorithms},
%	journal={computational complexity},
%	year={1992},
%	volume={2},
%	number={2},
%	pages={111-128},
%	mounth={June},
%	note = {\url{http://www.di.ens.fr/~fouque/ens-rennes/sac-LLL.pdf}}
%}
%
%@article{Merkle,
%	author = {Merkle, R. and Hellman, M.},
%	title = {Hiding Information and Signatures in Trapdoor Knapsacks},
%	journal = {IEEE Trans. Inf. Theor.},
%	issue_date = {1978},
%	volume = {24},
%	number = {5},
%	month = {Sep},
%	year = {1978},
%	issn = {0018-9448},
%	pages = {525--530},
%	numpages = {6},
%	url = {http://dx.doi.org/10.1109/TIT.1978.1055927},
%	doi = {10.1109/TIT.1978.1055927},
%	acmid = {2269393},
%	publisher = {IEEE Press},
%	address = {Piscataway, NJ, USA},
%} 

\section{Conclusion et perspectives}


\paragraph{} 
\justify 
Nous sommes arrivés au terme de ce voyage dans le monde de la sécurité web au travers de l'étude de DVWA mais nous n'avons fait qu'égratiner le sommet de la partie émergée de l'iceberg. On retiendra cependant les 3 types d'attaques les plus dangereuses : \textbf{ les injections, l'authentification et le XSS} ainsi que deux principes généraux : le filtrage et la configuration. D'une part, il est nécessaire de \textbf{filtrer les saisies des utilisateurs}, d'autre part il est essentiel de \textbf{configurer le serveur web utilisé} notamment pour qu'il soit le moins bavard possible et qu'il ne donne pas accès à des répertoires sensibles. On a, en effet, trop souvent tendance à laisser la configuration par défaut ! 

\paragraph{} De façon plus générale, qu'il soit question de développement web ou non, les techniques de programmation sécurisée et la veille technologique sont devenues pri\-mor\-dia\-les dans un monde essen\-tielle\-ment connecté et en évolution rapide. Une maîtrise complète de toutes les techniques reste cependant illusoire. C'est pourquoi, il est préférable d'utiliser des API, bibliothèques et framework web éprouvés et patchés.




\newpage
\appendix
%\include{annexes/annexe_A}
%\include{annexes/annexe_B}

\newpage
\nocite{*}  %affiche toutes les entrées du bib même celles qui ne sont pas citées.
% cf.    http://www.tuteurs.ens.fr/logiciels/latex/bibtex.html
% compilation en TROIS PHASE  bibtex traite un fichier *.aux mais bibtex mon_fichier comme bibtex mon_fichier.aux sont acceptés 
% latex mon_fichier.tex
% bibtex mon_fichier
% latex mon_fichier.tex


% \renewcommand{\bibname}{Toto}
% ou
\renewcommand{\refname}{Bibliographie}
% dans le préambule.
\bibliographystyle{alpha}
\bibliography{references}


\pagebreak
%\clearpage

\thispagestyle{empty}
\ThisTileWallPaper{1.45\paperwidth}{1.0\paperheight}{images/fret}


\addtolength{\wpXoffset}{-4.5cm}

\justify

%{\LARGE
%
%\color{white}{Ce rapport propose une analyse stratégique du fret aérien suiviant le modèle de Porter.}
%}

%\begin{figure}[!h]
%	\begin{center}
%		\label{}
%		\includegraphics[scale=0.4]{images/porter/porter_2}
%	\end{center}
%\end{figure}







\end{document}
