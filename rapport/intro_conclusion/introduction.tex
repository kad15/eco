
\pagebreak

\section{Introduction}

Le transport aérien de fret s'est développé à partir de 1911 où un
premier vol a consisté à livrer 15 kg de courrier entre deux villes asiatiques
distantes de 10 km sur une biplan Sommer. Depuis, le traffic a légèrement augmenté pour
atteindre en 2014, 51 millions de tonnes pour une valeur de US\$6.8 trillions \cite{RePEc:eee:jaitra:v:61:y:2017:i:c:p:34-40}.
Aujourd'hui, le fret aérien joue un rôle majeur dans l'économie et le développement international.

 
La présente étude vise à effectuer une analyse de marché suivant le modèle des 5 forces de Porter,
suivie d'une étude du rôle de l'Air Traffic Management (ATM) dans la stratégie des acteurs du secteur.
Pour terminer, on s'interrogera sur l'évolution future du rôle de l'ATM au regard de ces stratégies. 
   



