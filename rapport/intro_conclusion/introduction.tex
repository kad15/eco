
\pagebreak

\section*{Introduction}
\addcontentsline{toc}{section}{Introduction}
%\markboth{Introduction}{} 

Dans un monde où les communications électroniques prennent une importance incommensurable, la sécurité des messages échangés est un enjeu devenu majeur pour l'ensemble des parties prenantes, qu'il s'agisse d'entreprises, d'institutions ou de simples citoyens.

Le secteur de l'aéronautique n'est pas exempt de mettre en œuvre une sécurisation des nombreux messages échangés afin de parer au mieux tout acte malveillant dont le but serait de modifier des messages ou d'intercepter des informations sensibles. On peut aujourd'hui penser aux drones qui échangent avec le sol à la fois des instructions essentielles au vol et des données issues de la mission réalisée (mesures, prises de vue, etc\dots).

\paragraph{} La cryptographie donne les moyens à toutes les entités d'assurer la confidentialité, l'authenticité et l'intégrité des échanges. Un processus de chiffrement transforme le message \textit{en clair} en message \textit{chiffré}, incompréhensible, et le mécanisme de déchiffrement réalise l'opération inverse. L'idée sous-jacente de cette discipline est de faire en sorte qu'un message ayant subi un processus de chiffrement ne puisse être déchiffré qu'à l'aide d'un élément bien identifié par les parties prenantes, appelé \textit{clé}. Un \textit{cryptosystème} est défini par les mécanismes ou algorithmes de (dé)chiffrement, l'ensemble des textes en clair et textes chiffrés ainsi que les clés possibles. 
