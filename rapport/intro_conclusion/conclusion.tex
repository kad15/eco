\section{Conclusion et perspectives}

\paragraph{} Bien que le problème de somme de sous-ensembles (SSP) soit NP-complet, des algorithmes peuvent être mis en place pour tenter de casser le cryptosystème de Merkle-Hellman dans sa version basique. Si les algorithmes de force brute montrent très vite leurs limites, l'apport de théories plus ou moins récentes telles que la géométrie des nombres et la notion de réseau permet d'envisager des algorithmes nouveaux et élégants, bien que leur succès soit loin d'être garanti. Enfin, tous les sacs à dos ne se valent pas. Comme nous avons eu l'occasion de le voir, certaines méthodes comme la programmation dynamique ont horreur des sacs à dos dilatés alors que la densité est l'ennemie de l'algorithme LLL. La diversité des méthodes de cryptanalyse et la difficulté de se protéger contre l'une sans s'exposer à une autre justifie pleinement l'abandon de ce cryptosystème au profit de systèmes réputés plus sûrs, du moins à l'heure actuelle, comme RSA. 

\paragraph{} Il n'en reste pas moins
que le SSP reste un problème intéressant à étudier sur le plan théorique et pratique. L'application de méthode hybrides associant les techniques de programmation dynamique et arborescentes
avec heuristiques ont fait l'objet de nombreuses publications. Enfin, appliquer des algorithmes endémiques au domaine de l'intelligence et de l'apprentissage artificielle, tels les « colonies de fourmis », les algorithmes génétiques ou les réseaux de neurones  
pourraient fournir des pistes plus innovantes.