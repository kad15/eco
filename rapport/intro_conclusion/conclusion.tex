\section{Conclusion et perspectives}


\paragraph{} 
\justify 
Nous sommes arrivés au terme de ce voyage dans le monde de la sécurité web au travers de l'étude de DVWA mais nous n'avons fait qu'égratiner le sommet de la partie émergée de l'iceberg. On retiendra cependant les 3 types d'attaques les plus dangereuses : \textbf{ les injections, l'authentification et le XSS} ainsi que deux principes généraux : le filtrage et la configuration. D'une part, il est nécessaire de \textbf{filtrer les saisies des utilisateurs}, d'autre part il est essentiel de \textbf{configurer le serveur web utilisé} notamment pour qu'il soit le moins bavard possible et qu'il ne donne pas accès à des répertoires sensibles. On a, en effet, trop souvent tendance à laisser la configuration par défaut ! 

\paragraph{} De façon plus générale, qu'il soit question de développement web ou non, les techniques de programmation sécurisée et la veille technologique sont devenues pri\-mor\-dia\-les dans un monde essen\-tielle\-ment connecté et en évolution rapide. Une maîtrise complète de toutes les techniques reste cependant illusoire. C'est pourquoi, il est préférable d'utiliser des API, bibliothèques et framework web éprouvés et patchés.

