\section{Conclusion et perspectives}


\paragraph{} 
\justify 
Nous sommes arrivés au terme de ce voyage dans le monde de la sécurité web au travers de l'étude de DVWA mais nous n'avons fait qu'égratiner le sommet de la partie émergée de l'iceberg. On retiendra cependant les 3 types d'attaques les plus dangereuses : \textbf{ les injections, l'authentification et le XSS} et deux principes généraux : toujours \textbf{filtrer les saisies des utilisateurs}, par exemple en utilisant les expressions rationnelles ou une API et penser à \textbf{configurer le serveur web utilisé} notamment pour qu'il soit le moins bavard possible et qu'il ne donne pas accès à des répertoires sensibles.  

\paragraph{} De façon plus générale, que ce soit dans le cadre du développement web ou dans tout autre langage, les techniques de programmation sécurisée et la veille technologique sont devenues primordia-les dans un monde essentiellement connecté et en évolution rapide. Une maîtrise complète de toutes les techniques reste cependant théorique. C'est pourquoi, il est préférable d'utiliser des API, bibliothèques et framework web éprouvés et patchés.

